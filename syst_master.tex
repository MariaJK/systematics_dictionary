% !TEX TS-program = pdflatexmk
\documentclass[
12pt, % Main document font size
letterpaper, % Paper type, use 'letterpaper' for US Letter paper
oneside, % One page layout (no page indentation)
%twoside, % Two page layout (page indentation for binding and different headers)
headinclude,footinclude, % Extra spacing for the header and footer
BCOR5mm, % Binding correction
]{scrartcl}

%%%%%%%%%%%%%%%%%%%%%%%%%%%%%%%%%%%%%%%%%
% Arsclassica Article
% Structure Specification File
%
% This file has been downloaded from:
% http://www.LaTeXTemplates.com
%
% Original author:
% Lorenzo Pantieri (http://www.lorenzopantieri.net) with extensive modifications by:
% Vel (vel@latextemplates.com)
%
% License:
% CC BY-NC-SA 3.0 (http://creativecommons.org/licenses/by-nc-sa/3.0/)
%
%%%%%%%%%%%%%%%%%%%%%%%%%%%%%%%%%%%%%%%%%

%----------------------------------------------------------------------------------------
%	REQUIRED PACKAGES
%----------------------------------------------------------------------------------------

\usepackage[
nochapters, % Turn off chapters since this is an article        
beramono, % Use the Bera Mono font for monospaced text (\texttt)
eulermath,% Use the Euler font for mathematics
pdfspacing, % Makes use of pdftex’ letter spacing capabilities via the microtype package
dottedtoc % Dotted lines leading to the page numbers in the table of contents
]{classicthesis} % The layout is based on the Classic Thesis style

\usepackage{arsclassica} % Modifies the Classic Thesis package
\usepackage[T1]{fontenc} % Use 8-bit encoding that has 256 glyphs
\usepackage[utf8]{inputenc} % Required for including letters with accents

\usepackage{graphicx} % Required for including images
\graphicspath{{Figures/}} % Set the default folder for images

\usepackage{enumitem} % Required for manipulating the whitespace between and within lists
\usepackage{lipsum} % Used for inserting dummy 'Lorem ipsum' text into the template
\usepackage{subfig} % Required for creating figures with multiple parts (subfigures)
\usepackage{amsmath,amssymb,amsthm} % For including math equations, theorems, symbols, etc
\usepackage{varioref} % More descriptive referencing

%----------------------------------------------------------------------------------------
%	THEOREM STYLES
%---------------------------------------------------------------------------------------

\theoremstyle{definition} % Define theorem styles here based on the definition style (used for definitions and examples)
\newtheorem{definition}{Definition}

\theoremstyle{plain} % Define theorem styles here based on the plain style (used for theorems, lemmas, propositions)
\newtheorem{theorem}{Theorem}

\theoremstyle{remark} % Define theorem styles here based on the remark style (used for remarks and notes)

%----------------------------------------------------------------------------------------
%	HYPERLINKS
%---------------------------------------------------------------------------------------

\hypersetup{
%draft, % Uncomment to remove all links (useful for printing in black and white)
colorlinks=true, breaklinks=true, bookmarks=true,bookmarksnumbered,
urlcolor=webbrown, linkcolor=RoyalBlue, citecolor=webgreen, % Link colors
pdftitle={}, % PDF title
pdfauthor={\textcopyright}, % PDF Author
pdfsubject={}, % PDF Subject
pdfkeywords={}, % PDF Keywords
pdfcreator={pdfLaTeX}, % PDF Creator
pdfproducer={LaTeX with hyperref and ClassicThesis} % PDF producer
} % Include the structure.tex file which specified the document structure and layout
\usepackage[top=0.75in, bottom=0.75in, left=0.75in, right=0.75in]{geometry}
\usepackage{hyperref}
\newcommand{\blue}[1]{\textcolor{blue}{#1}}

\hyphenation{Fortran hy-phen-ation} % Specify custom hyphenation points in words with dashes where you would like hyphenation to occur, or alternatively, don't put any dashes in a word to stop hyphenation altogether

%----------------------------------------------------------------------------------------
%	TITLE AND AUTHOR(S)
%----------------------------------------------------------------------------------------

\title{\normalfont\spacedallcaps{Systematic Dictionary}} 
\author{} 
\date{} % An optional date to appear under the author(s)

%----------------------------------------------------------------------------------------

\begin{document}

%----------------------------------------------------------------------------------------
%	HEADERS
%----------------------------------------------------------------------------------------

\renewcommand{\sectionmark}[1]{\markright{\spacedlowsmallcaps{#1}}} % The header for all pages (oneside) or for even pages (twoside)
%\renewcommand{\subsectionmark}[1]{\markright{\thesubsection~#1}} % Uncomment when using the twoside option - this modifies the header on odd pages
\lehead{\mbox{\llap{\small\thepage\kern1em\color{halfgray} \vline}\color{halfgray}\hspace{0.5em}\rightmark\hfil}} % The header style

\pagestyle{scrheadings} % Enable the headers specified in this block

%----------------------------------------------------------------------------------------
%	TABLE OF CONTENTS & LISTS OF FIGURES AND TABLES
%----------------------------------------------------------------------------------------

\maketitle % Print the title/author/date block

\setcounter{tocdepth}{2} % Set the depth of the table of contents to show sections and subsections only

\tableofcontents % Print the table of contents

\listoffigures % Print the list of figures

\listoftables % Print the list of tables

%----------------------------------------------------------------------------------------
%	ABSTRACT
%----------------------------------------------------------------------------------------

\section*{Abstract} % This section will not appear in the table of contents due to the star (\section*)
A summary of systematics.

%----------------------------------------------------------------------------------------
\newpage % Start the article content on the second page, remove this if you have a longer abstract that goes onto the second page

%----------------------------------------------------------------------------------------
%	Input systematic files (Put your input here)
%----------------------------------------------------------------------------------------

\section{Introduction}
The signal timestream registering in our detectors can be approximated by the following expression
\begin{equation}
\begin{split}
d _i &= K \ast \left( n_i + g_i \int d \nu A_\mathrm{e} (\nu) F(\nu) \int d\Omega P (\theta _i,\phi _i) \right. \\ 
&\times [I(\theta _i, \phi _i) + \left. \gamma _i (Q(\theta _i,\phi _i)\cos (2\psi _i) + U(\theta _i,\phi _i) \sin (2\psi _i) ]  \vphantom{\int } \right) + \tilde{n}
\end{split}
\end{equation}
where $K \ast$ represents a convolution with the detector time response, $n_i$ is the noise, which we assume is uncorrelated with signal, $A_{\mathrm{e}} (\nu)$ represents the effective area of the telescope, $F(\nu)$ is the spectral responsivity, and $\tilde{n}_i$ represents noise terms that are not convolved by the detector response, including readout noise. 

The above expression is in many ways incomplete. For example, it does suggest that the Stokes $I$, $Q$, and $U$ parameters are frequency independent, which is certainly incorrect. The sky signal is generally composed of astrophysical signals with varying frequency dependence. This includes the CMB itself, thermal emission from dust, and synchrotron radiation. A CMB telescopes will observe the sky convolved with its beam function, $P(\theta, \phi)$.

\section{Optics}
\subsection{Systematic/Property Name}

\paragraph{Description:}
Description of systematic effect, including relevant equations and
parameterization for TWGs. Note that each variable in each equation should be
defined. This should include where we expect to get the value of this variable
from (TWG, literature, etc.)

\paragraph{Plan to model and/or measure:}
Plan to model/measure effect

\paragraph{Uncertainty/Range:}
This section should include the uncertainty of
known parameters and/or the expected range of parameters for consideration

\paragraph{Parameterization:}
This section should include the parameterization of figures of
merit and the output to the SWGs.

\subsection{Beam ellipticity}

\paragraph{Description:}
Many CMB experiments are designed to have angular sensitivity that can be described by an azimuthally symmetric two-dimensional Gaussian function
\begin{equation} 
P (\mathbf{x}) \propto \exp (-\mathbf{x} ^2/2\sigma ^2),
\end{equation}
where $\sigma$ represents the width of the beam. Optical aberrations will lead to asymmetries in the angular sensitivity which can often be captured by assuming that the Gaussian beam width is different along the two axis of a Cartesian coordinate system centered on the peak response
\begin{equation}
P (x,y) = \exp (-\frac{1}{2}[x^2/\sigma ^2_x + y^2/\sigma ^2_y]).
\end{equation}
This is referred to as an elliptical Gaussian function. Let's try \cite{Takahashi2010}.

\paragraph{Plan to model and/or measure:}
Beam ellipticities are extracted from beam maps which are acquired through scanning of a terrestrial source placed in the far-field of the optical system or by observing astrophysical point-sources such as the planets in our solar system.

\paragraph{Uncertainty/Range:}
Insert text

\paragraph{Parameterization:}

\subsection{Beam cross-polar response}

\paragraph{Description:}
Insert text

\paragraph{Plan to model and/or measure:}
Insert text

\paragraph{Uncertainty/Range:}
Insert text

\paragraph{Parameterization:}
Insert text


\section{Spectral Response Function}


\section{Polarization Modulators}


\section{Time Constants}
Detector time-response features can be probed by scanning quickly over bright objects such as planets, but our ability to probe such features might be limited by the maximum scan speed of the telescope mount. TES bolomoters are instrinsicly quite fast ~ 1 ms time-response. Features due to variation in scan speed might be visible in maps with sufficient integration. Additionally, when using a HWP, changes in time constant can change the apparent polarization angle, which must be corrected for.



\section{Spurious Signal and Noise}


\section{Atmospheric Effects}

\section{Papers on systematics}

There exist a number of useful papers that discuss characterization of systematics and their potential effects on CMB experiments. The following list is by no means exhaustive, my apologies if I missed some critical items.
\\ \\
\textbf{Papers that discuss systematics in the context of particular experiments}
\begin{description}

\item[\href{https://arxiv.org/abs/0906.4069}{Takahashi et al.\ (2009)}] Characterization of the BICEP Telescope for High-Precision Cosmic Microwave Background Polarimetry
\begin{itemize}[noitemsep]
\item Description of paper content
\end{itemize}

\item[\href{https://arxiv.org/abs/1106.3087}{Fraisse et al.\ (2011)}] SPIDER: Probing the Early Universe with a Suborbital Polarimeter
\begin{itemize}[noitemsep]
\item In many ways a continuation of the work presented in Takahasi et al. Some additions include taking a look at polarized sidelobes.
\end{itemize}


\end{description}

\noindent \textbf{Papers that provide a general discussion of systematics associated with CMB experiments}
\begin{description}

\item[\href{https://arxiv.org/abs/1211.5734}{Keating, Shimon, Yadav (2012)}] Self-Calibration of CMB Polarization Experiments
\begin{itemize}[noitemsep]
\item This paper describes a method that calibrates the polarization angles by enforcing $\mathbf{EB}$ and $\mathbf{TB}$ to zero
\end{itemize}

\end{description}



 

%----------------------------------------------------------------------------------------
%	BIBLIOGRAPHY
%----------------------------------------------------------------------------------------

\renewcommand{\refname}{\spacedlowsmallcaps{References}} % For modifying the bibliography heading
\bibliographystyle{unsrt}
\bibliography{syst.bib} % The file containing the bibliography

%----------------------------------------------------------------------------------------

\end{document}

There exist a number of useful papers that discuss characterization of systematics and their potential effects on CMB experiments. The following list is by no means exhaustive.
\\ \\
\textbf{Papers that discuss systematics in the context of particular experiments}
\begin{description}

\item[\href{https://arxiv.org/abs/1502.00608}{BICEP2 Collaboration (2015)}] BICEP2 III: Instrumental Systematics
\begin{itemize}[noitemsep]
\item Instrument and systematic characterization for the BICEP2 experiment
\item Topics: beams, pointing, crosstalk, ghost beams, polarization angles, detector transfer functions, EMI, etc.
\end{itemize}

\item[\href{https://arxiv.org/abs/1403.2369}{\pb\ Collaboration (2014)}] A Measurement of the Cosmic Microwave Background B-Mode Polarization Power Spectrum at Sub-Degree Scales with \pb\
\begin{itemize}[noitemsep]
\item B-mode measurement by \pb. Sec. 7 describes in detail the systematic instrumental effects, and their evaluation.
\item Topics: Differential gain, gain drifts, beams, pointing, crosstalk, polarization angles, polarization efficiency, HWP dependency, etc.
\end{itemize}

\item[\href{https://arxiv.org/abs/1106.3087}{Fraisse et al.\ (2011)}] SPIDER: Probing the Early Universe with a Suborbital Polarimeter
\begin{itemize}[noitemsep]
\item In many ways a continuation of the work presented in Takahasi et al. Some additions include taking a look at polarized sidelobes.
\end{itemize}

\item[\href{https://arxiv.org/abs/0906.4069}{Takahashi et al.\ (2009)}] Characterization of the BICEP Telescope for High-Precision Cosmic Microwave Background Polarimetry
\begin{itemize}[noitemsep]
\item Description of paper content
\end{itemize}

\item[\href{https://arxiv.org/abs/1004.2595}{Rosset et al.\ (2010)}] Planck pre-launch status: High Frequency Instrument polarization calibration
\begin{itemize}[noitemsep]
\item Setting constraints on the relative calibration for gains, polarization efficiencies and polarization angles in the focal plane needed to achieve Planck goals in terms of T,E,B leakage.
\end{itemize}



\end{description}

\noindent \textbf{Papers that provide a general discussion of systematics associated with CMB experiments}
\begin{description}

\item[\href{https://arxiv.org/abs/1211.5734}{Keating, Shimon, Yadav (2012)}] Self-Calibration of CMB Polarization Experiments
\begin{itemize}[noitemsep]
\item This paper describes a method that calibrates the polarization angles by enforcing $\mathbf{EB}$ and $\mathbf{TB}$ to zero.
\item Topics: polarization angles
\end{itemize}

\item[\href{https://arxiv.org/abs/1003.0198}{Hanson, Lewis, Challinor (2010)}] Asymmetric Beams and CMB Statistical Anisotropy
\begin{itemize}[noitemsep]
\item Discussing mathematical tools that allow one to propagate beam asymmetries into effects on CMB power spectra.
\item Topics: beams
\end{itemize}

\item[\href{https://arxiv.org/abs/0709.1513v4}{Shimon et al.\ (2008)}] CMB Polarization Systematics Due to Beam Asymmetry: Impact on Inflationary Science
\begin{itemize}[noitemsep]
\item Topics: beams
\end{itemize}

\end{description}

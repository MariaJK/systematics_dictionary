There exist a number of useful papers that discuss characterization of systematics and their potential effects on CMB experiments. The following list is by no means exhaustive, my apologies if I missed some critical items.
\\ \\
\textbf{Papers that discuss systematics in the context of particular experiments}
\begin{description}

\item[\href{https://arxiv.org/abs/0906.4069}{Takahashi et al.\ (2009)}] Characterization of the BICEP Telescope for High-Precision Cosmic Microwave Background Polarimetry
\begin{itemize}[noitemsep]
\item Description of paper content
\end{itemize}

\item[\href{https://arxiv.org/abs/1106.3087}{Fraisse et al.\ (2011)}] SPIDER: Probing the Early Universe with a Suborbital Polarimeter
\begin{itemize}[noitemsep]
\item In many ways a continuation of the work presented in Takahasi et al. Some additions include taking a look at polarized sidelobes.
\end{itemize}


\end{description}

\noindent \textbf{Papers that provide a general discussion of systematics associated with CMB experiments}
\begin{description}

\item[\href{https://arxiv.org/abs/1211.5734}{Keating, Shimon, Yadav (2012)}] Self-Calibration of CMB Polarization Experiments
\begin{itemize}[noitemsep]
\item This paper describes a method that calibrates the polarization angles by enforcing $\mathbf{EB}$ and $\mathbf{TB}$ to zero
\end{itemize}

\end{description}
